% ------------------------------------------------------------------------
% Ígor Assis Rocha Yamamoto
% contato@igoryamamoto.com
% Modelo criado a partir da abnTeX2, em conformes com a ABNT NBR 14724:2011
% ------------------------------------------------------------------------

\documentclass[
	% -- opções da classe memoir --
	12pt,				% tamanho da fonte
	openright,			% capítulos começam em pág ímpar (insere página vazia caso preciso)
	twoside,			% para impressão em verso e anverso. Oposto a oneside
	a4paper,			% tamanho do papel. 
	% -- opções da classe abntex2 --
	%chapter=TITLE,		% títulos de capítulos convertidos em letras maiúsculas
	%section=TITLE,		% títulos de seções convertidos em letras maiúsculas
	%subsection=TITLE,	% títulos de subseções convertidos em letras maiúsculas
	%subsubsection=TITLE,% títulos de subsubseções convertidos em letras maiúsculas
	% -- opções do pacote babel --
	english,			% idioma adicional para hifenização
	brazil				% o último idioma é o principal do documento
	]{abntex2}

% ---
% Pacotes básicos 
% ---
%\usepackage{arial}
\usepackage[T1]{fontenc}		% Selecao de codigos de fonte.
\usepackage[utf8]{inputenc}
\usepackage{lastpage}			% Usado pela Ficha catalográfica
\usepackage{indentfirst}		% Indenta o primeiro parágrafo de cada seção.
\usepackage{color}				% Controle das cores
\usepackage{graphicx}			% Inclusão de gráficos
\usepackage{microtype} 			% para melhorias de justificação

% ---
% Pacotes adicionados
% ---
\usepackage{amsmath}
\usepackage{amssymb,amsfonts,amsthm}
\usepackage{setspace}
		
% ---
% Pacotes adicionais, usados apenas no âmbito do Modelo Canônico do abnteX2
% ---
% \usepackage{lipsum}				% para geração de dummy text
\usepackage{todonotes}

% ---
% Pacotes de citações
% ---
\usepackage[brazilian,hyperpageref]{backref}	 % Paginas com as citações na bibl
\usepackage[num]{abntex2cite}	% Citações padrão ABNT

\usepackage{cite}
\renewcommand\citeleft{[}
\renewcommand\citeright{]}

% --- 
% CONFIGURAÇÕES DE PACOTES
% --- 
\renewcommand{\bf}[1]{\mathbf{#1}}
\renewcommand{\rm}[1]{\mathrm{#1}}

% Configurações do pacote backref
\renewcommand{\backrefpagesname}{Citado na(s) página(s):~} % Usado sem a opção hyperpageref de backref
\renewcommand{\backref}{} % Texto padrão antes do número das páginas
\renewcommand*{\backrefalt}[4]{ % Define os textos da citação
	\ifcase #1 %
		Nenhuma citação no texto.%
	\or
		Citado na página #2.%
	\else
		Citado #1 vezes nas páginas #2.%
	\fi}%
	
% % ---
% Agradecimentos
% ---

\begin{capa}
\begin{center}
{\noindent\includegraphics[width=1\linewidth]{logo}}

\vspace{2.5cm}
{\noindent 
    {\bfseries \huge Desenvolvimento de Aplicativo para}\par
    {\bfseries \huge Conectar Academias e Treinadores}\par
    {\bfseries \huge Utilizando Testes Automatizados}\par
    {\bfseries \huge de Interface Gráfica}
} 
\vspace{3.2cm}

{\noindent {\itshape \large Relatório submetido à Universidade Federal de Santa Catarina}}

{\noindent {\itshape \large como requisito para a aprovação da disciplina:}}

{\noindent {\itshape \bfseries \large DAS 5511: Projeto de Fim de Curso}}
 
\vspace{3cm} 
  
{\noindent {\itshape \bfseries \large Ígor Assis Rocha Yamamoto}} 

\vspace{3.2cm}

\thispagestyle{empty}

{\noindent {\itshape Florianópolis, Outubro de 2018}} 

\end{center}
\end{capa}


% ---
% Configurações de aparência do PDF final
% ---

\definecolor{blue}{RGB}{41,5,195} % alterando o aspecto da cor azul

% ---
% Informações do PDF
% ---

\makeatletter

\hypersetup{
     	%pagebackref=true,
		colorlinks=true,       		% false: boxed links; true: colored links
    	linkcolor=blue,          	% color of internal links
    	citecolor=blue,        		% color of links to bibliography
    	filecolor=magenta,      		% color of file links
		urlcolor=blue,
		bookmarksdepth=4
}

\makeatother

% --- 
% Espaçamentos entre linhas e parágrafos 
% --- 

% O tamanho do parágrafo é dado por:
\setlength{\parindent}{1.3cm}

% Controle do espaçamento entre um parágrafo e outro:
\setlength{\parskip}{0.2cm}  % tente também \onelineskip

% ---
% Compila o indice
% ---
\makeindex

% ----------------------------------------------------------
% INÍCIO DO DOCUMENTO
% ----------------------------------------------------------
\begin{document}
\listoftodos
%\selectlanguage{english}
\selectlanguage{brazil}

\frenchspacing % Retira espaço extra obsoleto entre as frases.

% ----------------------------------------------------------
% ELEMENTOS PRÉ-TEXTUAIS
% ----------------------------------------------------------
\pretextual

% ---
% Agradecimentos
% ---

\begin{capa}
\begin{center}
{\noindent\includegraphics[width=1\linewidth]{logo}}

\vspace{2.5cm}
{\noindent 
    {\bfseries \huge Desenvolvimento de Aplicativo para}\par
    {\bfseries \huge Conectar Academias e Treinadores}\par
    {\bfseries \huge Utilizando Testes Automatizados}\par
    {\bfseries \huge de Interface Gráfica}
} 
\vspace{3.2cm}

{\noindent {\itshape \large Relatório submetido à Universidade Federal de Santa Catarina}}

{\noindent {\itshape \large como requisito para a aprovação da disciplina:}}

{\noindent {\itshape \bfseries \large DAS 5511: Projeto de Fim de Curso}}
 
\vspace{3cm} 
  
{\noindent {\itshape \bfseries \large Ígor Assis Rocha Yamamoto}} 

\vspace{3.2cm}

\thispagestyle{empty}

{\noindent {\itshape Florianópolis, Outubro de 2018}} 

\end{center}
\end{capa}

\begin{center}
{\noindent 
    {\bfseries \Large Desenvolvimento de aplicativo para}\par
    {\bfseries \Large locação de academias de musculação}\par
    {\bfseries \Large utilizando testes automatizados}\par
    {\bfseries \Large de interface gráfica}
} 
\vspace{1.5cm}

{\noindent {\itshape \bfseries \large Ígor Assis Rocha Yamamoto}}

\vspace{1cm}

%\begin{espacosimples}
{\noindent {\large Esta monografia foi julgada no contexto da disciplina}}

{\noindent {\bfseries \large DAS 5511: Projeto de Fim de Curso}}

{\noindent {\large e aprovada na sua forma final pelo}}

{\noindent {\bfseries \large Curso de Engenharia de Controle e Automação}}

\vspace{7cm}

{\noindent {\itshape \bfseries \large Prof. Rômulo Silva de Oliveira da UFSC}}

\vspace{1cm}

% UGLY FIX - FIND A DIFFERENT WAY TO DO THIS
{\noindent {\large \underline{\hspace{6cm}}}}
%/UGLY FIX

%\end{espacosimples}

\end{center}

\vspace{-0.2cm}


% \include{pre-textual/ficha}
% \include{pre-textual/errata}
% ---
% Inserir folha de aprovação
% ---

% Isto é um exemplo de Folha de aprovação, elemento obrigatório da NBR
% 14724/2011 (seção 4.2.1.3). Você pode utilizar este modelo até a aprovação
% do trabalho. Após isso, substitua todo o conteúdo deste arquivo por uma
% imagem da página assinada pela banca com o comando abaixo:
%
% \includepdf{folhadeaprovacao_final.pdf}
%
\begin{folhadeaprovacao}


\thispagestyle{empty}

{\large Banca Examinadora:}

\vspace{1.3cm}

\begin{flushright}

{\large Andrio Renan Gonzatti Frizon/Jungle Devs}

{\large Orientador na Empresa}

\vspace{1.2cm}

{\large Antonio Adalberto Duarte Júnior/Jungle Devs}

{\large Orientador na Empresa}

\vspace{1.2cm}
%\begin{espacosimples}
{\large Prof. Rômulo Silva de Oliveira}

{\large Orientador no Curso}
%\end{espacosimples}

\vspace{1.2cm}
 
%\begin{espacosimples}
{\large Prof. Hector Bessa Silveira}

{\large Responsável pela disciplina}
%\end{espacosimples}

\vspace{1cm}

{\large Prof. Eduardo Camponogara, Avaliador}

\vspace{0.8cm}

{\large Matheus Felipe Souza Valin, Debatedor}

\vspace{0.8cm}

{\large Rafael Scheffer, Debatedor}

\end{flushright}
  
\end{folhadeaprovacao}
% \include{pre-textual/dedicatoria}
% ---
% Agradecimentos
% ---

\begin{agradecimentos}
Agradeço a Thais

\end{agradecimentos}

% \include{pre-textual/epigrafe}
% ---
% RESUMOS
% ---

% resumo em português
\setlength{\absparsep}{18pt} % ajusta o espaçamento dos parágrafos do resumo
\begin{resumo}
Descrição geral da empresa (natureza, mercado, processos, etc.), problema-foco atacado no PFC, o que foi feito, principais resultados atingidos, etc.
Se o documento for escrito em outra língua que não o Português, então é necessário fazer um Resumo Estendido em Português, ao invés deste resumo enxuto.


 \textbf{Palavras-chave}:.
\end{resumo}

% resumo em inglês
\begin{resumo}[Abstract]
 \begin{otherlanguage*}{english}
Resumo em Inglês
   \vspace{\onelineskip}
 
   \noindent 
   \textbf{Keywords}:Palavras Chaves.
 \end{otherlanguage*}
\end{resumo}

%% resumo em francês 
%\begin{resumo}[Résumé]
% \begin{otherlanguage*}{french}
%    Il s'agit d'un résumé en français.
% 
%   \textbf{Mots-clés}: latex. abntex. publication de textes.
% \end{otherlanguage*}
%\end{resumo}
%
%% resumo em espanhol
%\begin{resumo}[Resumen]
% \begin{otherlanguage*}{spanish}
%   Este es el resumen en español.
%  
%   \textbf{Palabras clave}: latex. abntex. publicación de textos.
% \end{otherlanguage*}
%\end{resumo}
% ---


% inserir lista de ilustrações
\pdfbookmark[0]{\listfigurename}{lof}
\listoffigures*
\cleardoublepage

% inserir lista de tabelas
\pdfbookmark[0]{\listtablename}{lot}
\listoftables*
\cleardoublepage

% inserir lista de siglas
% ---
% inserir lista de abreviaturas e siglas
% ---

\begin{siglas}
 \item[API] Application Programming Interface
 \item[HTTP] Hypertext Transfer Protocol
 \item[JSON] JavaScript Object Notation
 \item[MVC] Model View Controller
 \item[REST] Representational State Transfer
 \item[UI] User Interface
\end{siglas}

%\begin{simbolos}
%  \item[$ CO_2 $] Dióxido de Carbono
%  \item[$C_{3+}$] Hidrocarbonetos com três ou mais carbonos
%\end{simbolos}


% inserir o sumario
\pdfbookmark[0]{\contentsname}{toc}
\tableofcontents*
\cleardoublepage

% ----------------------------------------------------------
% ELEMENTOS TEXTUAIS
% ----------------------------------------------------------
\textual
\chapter{Introdução}

\section{Motivação}

A demanda por soluções tecnológicas na área da saúde vem crescendo a nível global, com os  \textit{smartphones} ganhando destaque entre as ferramentas utilizadas \todo{citar mhealth-who}. O termo \textit{mHealth} (do inglês, \textit{mobile health}) é utilizado para o estudo da prática da medicina e saúde pública através do uso da tecnologia móvel (\textit{mobile} em inglês). Diversos tipos de segmentos da área da saúde, como assistência médica, nutrição, gestão de estilo de vida e bem-estar compõem um mercado de aplicações de \textit{mHealth} que já ultrapassa o valor de mercado de $25$ bilhões de dólares anuais, com mais de $325.000$ aplicativos disponíveis, somando um total de $3.6$ bilhões de downloads nas lojas virtuais (estimativa de 2017).\todo{citar mhealth-economics}

O segmento da área da saúde que mais se destaca no uso de tecnologias móveis, obtendo o maior número de assinaturas de usuários e receita total\todo{citar mhealth-market}, é o de atividades físicas (\textit{fitness} em inglês). Tendo em vista este fator, a Jungle Devs, empresa onde foi realizado este trabalho de PFC, foi contratada por um cliente da Austrália para desenvolver um aplicativo móvel envolvendo problemas dentro deste segmento. A ideia do projeto é desenvolver aplicativos para as plataformas \textit{iOS} e \textit{Android} que facilitem a conexão entre academias de musculação e treinadores pessoais profissionais, permitindo que aquelas sejam locadas por estes.

O uso da tecnologia móvel de forma semelhante ao proposto - através de uma abordagem de economia colaborativa (\textit{sharing economy} em inglês) - já obteve sucesso em outros campos de atuação \todo{citar sharing-economy}. Exemplos como \textit{Airbnb}, no ramo imobiliário, e \textit{Uber}, no setor de transportes, mostram como indivíduos e grupos podem se beneficiar com o uso de recursos subutilizados, oferecendo os mesmos na forma de serviço. Oportunidades para usufruir deste benefício da economia colaborativa foram detectadas na proposta do aplicativo estudado neste PFC, a exemplo de academias e locais com infraestrutura disponível para musculação (hotéis, condomínios, entre outros) que poderiam otimizar o uso de seus espaços e recursos, tendo em vista a demanda existente da população por exercícios físico.

Em face ao cenário apresentado, a construção do aplicativo adotou a abordagem de desenvolvimento ágil (\textit{agile} em inglês), com o uso de técnicas como o \textit{Scrum} para assegurar a rápida prototipação da solução, permitindo a iteração de desenvolvimento frequente baseada no \textit{feedback} do cliente de forma que o aplicativo atenda as necessidades reais dos usuários. Além de selecionar e fazer uso de ferramentas de programação modernas 

\section{Objetivos}-

- Desenvolvimento do aplicativo iOS
    - Interface gráfica
    - Integração com sistemas
    
- Estudo, pesquisa e implementação de testes automatizados de interface 
    - Pesquisa de ferramentas
    - Experimentos
    - Estudo de viabilidade

\section{Estrutura do Documento}

\todo{explicar estrutura do doc}
Este trabalho está dividido da seguinte forma: 

No capítulo 2, a contextualização

No capítulo 3, os conceitos básicos

No capítulo 4, a descrição conceitual

No capítulo 5, o desenvolvimento

No capítulo 6, os resultados

E por fim, no capítulo 7, as conclusoes 

\chapter{Contextualização}

\section{A Empresa}
\missingfigure{logo da Jungle}
A Jungle Devs \todo{referenciar logo-jungledevs} é uma empresa do setor de TI fundada em 2018 por ex-alunos do curso de Engenharia de Controle e Automação da UFSC, que trabalhavam juntos desde 2010. Sediada em Florianópolis, a empresa atua na área de desenvolvimento de software, com especialidade em produtos web e \textit{mobile}, e oferece programas de capacitação para futuros profissionais do setor tecnológico.

A empresa foi oficialmente fundada em fevereiro de 2018 por três engenheiros. Antonio, Andrio e Vinicius moravam e trabalhavam juntos em outra empresa do setor de TI, desenvolvendo produtos de tecnologia. Os três decidiram fundar a própria empresa com o conhecimento técnico adquirido pelos membros durante dois anos de trabalho na antiga empresa, somado aos mais de dez anos de experiência de Antonio na indústria de TI, que resultaram em forte rede de conexões com profissionais ao redor do mundo, em especial na África do Sul, Polônia e Vale do Silício, na Califórnia - Estados Unidos. A ideia de abrir o próprio negócio surgiu da identificação da necessidade por novos profissionais que o setor de tecnologia apresenta. Tendo em vista a possibilidade de capacitar pessoas fornecendo experiência prática com os projetos emergentes das parcerias profissionais, os três engenheiros decidiram criar a empresa com o foco no desenvolvimento de pessoas.

A partir da rede de conexões ao redor do mundo, empresa teve início com o desenvolvimento de projetos de aplicações \textit{web} e \textit{mobile} para clientes de diversos países, como Estados Unidos e Austrália. A proposta de capacitar novos profissionais de tecnologia pôde ser validada com outros três engenheiros, agora também sócios da Jungle Devs. Henrique, Pedro e Rodrigo ingressaram na empresa sem nenhuma experiência profissional prévia no desenvolvimento de produtos de tecnologia, e a partir do envolvimento dos três diretamente com o desenvolvimento de novas aplicações, formou-se um time. Com o feedback positivo dos parceiros e, também dos clientes, novos projetos começaram a surgir e, também, uma nova demanda por profissionais capacitados.

A Jungle Devs, então, surgiu com a missão de repensar a forma como pessoas aprendem e criam tecnologia, dando o início desse processo com um programa de capacitação denominado \textit{Academy}. Os participantes do \textit{Academy} passam por diferentes períodos de treinamento, partindo do aprendizado de conceitos básicos de programação até a atuação no desenvolvimento de um projeto real e conexão do novo profissional com o mercado de trabalho. Este trabalho de PFC está inserido na última etapa do programa com foco em desenvolvimento \textit{mobile}.

Em paralelo ao propósito de fomentar o mercado de trabalho do setor de tecnologia, a Jungle Devs oferece à empresas os serviços de idealização de produtos, design, desenvolvimento, revisão e testes, lançamento de produtos no mercado e serviços de manutenção. Assim, estes serviços são utilizados como plataforma de aprendizado para novos profissionais, tendo o suporte de um time de engenharia próprio da empresa.

Atualmente a empresa tem crescido organicamente a partir de indicações, tanto em número de pessoas como de projetos. De fevereiro à novembro de 2018, a empresa teve crescimento do número de pessoas de 7 para 13, somando participantes do \textit{Academy} e time de engenharia. Além disso, a empresa conta com uma rede de profissionais denominada \textit{Jungle Community} que atuam como freelancers nos projetos da empresa. Os projetos são em maioria desenvolvidos para clientes do exterior, contemplando países da América, África, Europa, Ásia e Oceania. O surgimento de novos projetos tem sido recorrente com a recomendação de novos clientes a partir dos atuais e anteriores, sendo este um bom indicador de qualidade da entrega dos produtos.

\section{A Aplicação}
A partir de uma recomendação de um cliente de outro projeto de software desenvolvido na Austrália, a Jungle Devs foi contratada para desenvolver um aplicativo. A aplicação, denominada Gyymi, tem o intuito de aproximar treinadores pessoais profissionais com academias de musculação. O cliente australiano identificou uma oportunidade de mercado com a ideia de um produto que pudesse tirar proveito da ociosidade de muitos estabelecimentos com infraestrutura para treinamentos de musculação, tanto de academias que já oferecem este serviço, mas também de locais que potencialmente poderiam oferecê-lo (como hotéis, condomínios, entre outros), principalmente os localizados em países desenvolvidos, como é o caso da Austrália.

O uso da tecnologia móvel de forma semelhante ao proposto pelo aplicativo - através de uma abordagem de economia colaborativa (\textit{sharing economy} em inglês) - já obteve sucesso em outros campos de atuação, como no setor de transportes e no ramo imobiliário \todo{citar sharing-economy}. O modelo de estratégia proposto permite que indivíduos e grupos se beneficiem com o uso de recursos subutilizados, oferecendo os mesmos na forma de serviço. Da mesma maneira, o Gyymi tem o potencial de oferecer a oportunidade de estabelecimentos voltados à atividades físicas otimizarem o uso de seus espaços e recursos, tendo em vista a demanda crescente da população mundial por exercícios físicos \todo{provar essa bagaça}.

O desenvolvimento do produto ocorrerá em diversas etapas. Inicialmente foi lançada uma versão piloto do aplicativo apenas no sistema \textit{iOS}, discutida dentro deste trabalho. Futuramente, uma versão para a plataforma \textit{Android} será lançada, aplicativo que também já está em fase de desenvolvimento. A versão piloto está em atual fase de testes com parceiros de estabelecimentos localizados em Sydney, na Austrália. Para o ano seguinte, está previsto o lançamento da primeira versão oficial do aplicativo.

\chapter{Conceitos Básicos}

\section{Métodos Ágeis de Desenvolvimento de Software}

\subsection{Agile}

\subsection{Scrum}

\section{Desenvolvimento de Aplicativos iOS}

\subsection{Sistema Operacional e Arquitetura da Plataforma}

\subsection{Ambiente de Desenvolvimento}

\subsection{Swift}

\subsection{Gerenciamento de Dependências}

\section{Versionamento de Código}

\section{RESTful API's}

\section{Testes Automatizados}
\missingfigure{Figuras classicas dos testes, modelo em V, piramide etc}
\subsection{Testes de UI}

\chapter{Gyymi}
Neste capítulo é apresentado o projeto do aplicativo Gyymi do ponto de vista do usuário do sistema. Os principais fluxos de uso do aplicativo são expostos, com comentários da parte técnico quando necessários.

% ********
% Cadastro
% ********
\section{Cadastro de Usuários e Estabelecimentos}
Ao entrar acessar o aplicativo pela primeira vez, o usuário encontra a tela de entrada (Figura \ref{fig:landing}). Nesta, três ações podem ser tomadas: cadastrar um estabelecimento (opção "I am a gym"), cadastrar um perfil de treinador (opção "I am a trainer") ou realizar o login na plataforma caso já tenha uma conta (opção "Login to existing account"). A seguir são detalhados os dois fluxos de cadastro do aplicativo.
\begin{figure}[H]
    \centering
    \includegraphics[width=0.4\textwidth]{pfc/figuras/landing.png}
    \caption{Tela de entrada do aplicativo}
    \label{fig:landing}
\end{figure}

% ******************
% Cadastro academias
% ******************
\subsection{Academias} \label{sec:register-gym}
Selecionada a opção por cadastro de um estabelecimento, o usuário primeiramente deve cadastrar os dados do administrador do local (ver Figura \ref{fig:register-manager-data}). Os seguintes dados são solicitados: primeiro nome, último nome, número do celular com código de área, e-mail (com campo de verificação) e senha (com campo de verificação). Para prosseguir com o cadastro, o usuário deve preencher os campos com dados válidos (caso contrário, alertas de erro são apresentados na tela). Ao clicar o botão "Next" uma chamada de API é feita ao back-end passando os dados digitados como parâmetro. Em caso de sucesso, a próxima tela do cadastro é apresentada; em caso de erro (como um e-mail de usuário já cadastrado), um alerta de erro é apresentado - Figura \ref{fig:register-manager-error}.

\begin{figure}[H]
	\centering
    \begin{subfigure}[b]{0.4\textwidth}
        \includegraphics[width=\textwidth]{pfc/figuras/register-manager.png}
        \caption{Dados do administrador}
        \label{fig:register-manager-data}
    \end{subfigure}
    ~
	\begin{subfigure}[b]{0.4\textwidth}
        \includegraphics[width=\textwidth]{pfc/figuras/register-manager-error.png}
        \caption{Alerta de erro}
        \label{fig:register-manager-error}
    \end{subfigure}
    ~
    \caption{Tela de cadastro do usuário - administrador da academia}
    \label{fig:register-manager}
\end{figure}

Após o cadastro do usuário administrador da academia ter sido realizado com sucesso, o perfil do estabelecimento é cadastrado. Três telas fazem parte deste fluxo (ver Figura \ref{fig:register-gym}): primeiro informações gerais da academia (nome, endereço, número para contato com o estabelecimento, web-site e endereço de mídias sociais) devem ser passadas - Figura \ref{fig:register-gym-info}; em seguida um tela (Figura \ref{fig:register-gym-amenities}) com opções selecionáveis de facilidades e equipamentos fornecidos pela academia é apresentada; por último, o usuário tem a opção de carregar fotos do estabelecimento (Figura \ref{fig:register-gym-photos}), selecionando uma delas para ser exibida como foto de capa (a ser mostrada no perfil da academia).

\begin{figure}[H]
	\centering
    \begin{subfigure}[b]{0.3\textwidth}
        \includegraphics[width=\textwidth]{pfc/figuras/register-gym-info.png}
        \caption{Registro das informações gerais da academia}
        \label{fig:register-gym-info}
    \end{subfigure}
    ~
	\begin{subfigure}[b]{0.3\textwidth}
        \includegraphics[width=\textwidth]{pfc/figuras/register-gym-amenities.png}
        \caption{Registro das facilidades e equipamentos}
        \label{fig:register-gym-amenities}
    \end{subfigure}
    ~
    \begin{subfigure}[b]{0.3\textwidth}
        \includegraphics[width=\textwidth]{pfc/figuras/register-gym-photos.png}
        \caption{Registro das fotos do local e da logo}
        \label{fig:register-gym-photos}
    \end{subfigure}
    ~
    \caption{Telas de cadastro de um estabelecimento}
    \label{fig:register-gym}
\end{figure}

Ao término do cadastro do estabelecimento, uma tela de boas vindas (Figura \ref{fig:gym-welcome}) é apresentada ao usuário. A tela apresenta uma marcação (gota de suor branca) da academia em um mapa no local real cadastrado anteriormente (mapa obtido através do serviço do sistema de geolocalização, que é apresentado no próximo capítulo) e outros locais já cadastrados na plataforma são demarcados também (gotas pretas de suor menores), informação proveniente do back-end. A tela apresenta duas opções de ação: registrar detalhes da conta bancária (opção "Add bank details", tela não implementada para a versão piloto do aplicativo) e uma opção para pular o cadastro da conta (opção "Skip"). Ambas as ações levam o usuário as telas de uso da academia, que são apresentadas nas próximas secções.

\begin{figure}[ht]
    \centering
    \includegraphics[width=0.4\textwidth]{pfc/figuras/gym-welcome.png}
    \caption{Tela de boas vindas para o estabelecimento}
    \label{fig:gym-welcome}
\end{figure}

% ********************
% Cadastro Treinadores
% ********************
\subsection{Treinadores}
Caso o usuário selecione a opção de cadastro de um treinador na tela de entrada, o mesmo é redirecionado para a tela de cadastro de informações gerais de usuário (Figura \ref{fig:register-trainer-info}), com os seguintes campos de dados: primeiro nome, último nome, e-mail, número do celular, senha (com campo de verificação). Caso as informações preenchidas sejam válidas, o cadastro prossegue para a próxima tela; caso contrário, alertas de erros são exibidos (como no caso do estabelecimento - Secção \ref{sec:register-gym})

Após o preenchimento dos campos, uma requisição para registrar os dados na plataforma é feita para o back-end, o qual faz uma requisição ao serviço de envio de SMS com os dados do número do celular do usuário. Este serviço então envia um SMS com um código de verificação, que deve ser preenchido no campo "Verification Code" da tela de verificação por SMS (ver Figura \ref{fig:register-trainer-verification}). Nesta tela o usuário tem duas opções de ação: verificar o código preenchido (opção que dispara uma nova requisição ao back-end para identificar se o código fornecido é o mesmo gerado pelo serviço de SMS) ou reenviar o código de verificação (opção que desencadeia um novo envio de SMS para o usuário através de nova solicitação para tal ao serviço de SMS).

\begin{figure}[H]
	\centering
    \begin{subfigure}[b]{0.4\textwidth}
        \includegraphics[width=\textwidth]{pfc/figuras/register-trainer.png}
        \caption{Dados do treinador}
        \label{fig:register-trainer-info}
    \end{subfigure}
    ~
	\begin{subfigure}[b]{0.4\textwidth}
        \includegraphics[width=\textwidth]{pfc/figuras/register-trainer-verification.png}
        \caption{Verificação por SMS}
        \label{fig:register-trainer-verification}
    \end{subfigure}
    ~
    \caption{Tela de cadastro do usuário - treinador pessoal}
    \label{fig:register-trainer}
\end{figure}

Uma vez que o código de verificação é validado, o fluxo prossegue para uma tela de boas vindas (Figura \ref{fig:tr-welcome}). Nesta tela, o usuário tem a ação de configurar seu perfil de treinador (opção "Set up your trainer profile") ou a ação de pular esta etapa (opção "Skip"). Caso selecionada a opção de configuração, o cadastro prossegue; caso selecionada a outra opção, o usuário é direcionado a tela principal do treinador.

\begin{figure}[H]
    \centering
    \includegraphics[width=0.4\textwidth]{pfc/figuras/tr-congratulations.png}
    \caption{Tela de boas vindas para o treinador}
    \label{fig:tr-welcome}
\end{figure}

A configuração de perfil de treinador segue um fluxo de três telas (Figura \ref{fig:register-tr-profile}). Primeiro, o usuário deve registrar as informações gerais do perfil (Figura \ref{fig:register-tr-profile-info}), informando de forma opcional os campos: mantra, área de treino, endereço de mídias sociais e web-site. Em seguida, o usuário é direcionado a uma tela com opções selecionáveis de habilidades e especialidades (Figura \ref{fig:register-tr-skills}) que ele pode oferecer em seus treinos físicos. Por último, uma tela de confirmação do cadastro do perfil é exibida (Figura \ref{fig:register-tr-profile-confirmation}), com opções de edição caso o usuário deseje alterar alguma informação. Ao pressionar o botão de confirmar, uma requisição é feita ao back-end passando os dados de perfil do treinador para registro no sistema. Após a requisição ter obtido sucesso, o usuário é direcionado a tela principal de uso do aplicativo do treinador, a ser apresentada nas próximas secções.

\begin{figure}[H]
	\centering
    \begin{subfigure}[b]{0.3\textwidth}
        \includegraphics[width=\textwidth]{pfc/figuras/tr-register-profile-1.png}
        \caption{Registro das informações gerais do treinador}
        \label{fig:register-tr-profile-info}
    \end{subfigure}
    ~
	\begin{subfigure}[b]{0.3\textwidth}
        \includegraphics[width=\textwidth]{pfc/figuras/tr-register-profile-2.png}
        \caption{Registro das habilidades e especialidades}
        \label{fig:register-tr-skills}
    \end{subfigure}
    ~
    \begin{subfigure}[b]{0.3\textwidth}
        \includegraphics[width=\textwidth]{pfc/figuras/tr-register-profile-3.png}
        \caption{Confirmação do registro de perfil}
        \label{fig:register-tr-profile-confirmation}
    \end{subfigure}
    ~
    \caption{Fluxo de cadastro de perfil de treinador}
    \label{fig:register-tr-profile}
\end{figure}

% *******************
% Interface academias
% *******************
\section{Interface para Academias}
Nesta secção, é apresentada a interface para usuários que cadastraram um estabelecimento na plataforma. As principais funcionalidades implementadas para a versão piloto do aplicativo são abordadas: dashboard com dados semanais de uso da academia, calendário de sessões agendadas e configuração de agenda semanal para locação, perfil da academia.

Na interface para as academias o usuário tem a opção de navegar por cinco telas principais, selecionadas a partir da barra de navegação na parte inferior do aplicativo (Figura \ref{fig:gym-tabbar}). Da esquerda para a direita, as opções de navegação são: dashboard da academia; treinadores com sessões agendadas (não implementado para o piloto); calendário de sessões agendadas; central de notificações (não implementado para o piloto); perfil do estabelecimento.

Todos os dados que aparecem nas telas são provenientes de chamadas de API feitas ao back-end da aplicação. De acordo com o endpoint acessado, o back-end realiza o acesso ao banco de dados remoto da aplicação, efetua cálculos e aplica as lógicas de negócio se necessário. Em seguida, o mesmo envia a resposta à requisição. A chamada geralmente é feita no momento em que os componentes da tela são carregados ou imediatamente antes de serem renderizados. Mais detalhes desta integração são discutidos no próximo capítulo.

\begin{figure}[H]
    \centering
    \includegraphics{pfc/figuras/gym-tabbar.png}
    \caption{Barra de navegação da interface para academias}
    \label{fig:gym-tabbar}
\end{figure}


\subsection{Dashboard da Academia}
A tela de dashboard da academia (Figura \ref{fig:gym-dashboard}) apresenta dados de uso e de receita semanais do estabelecimento. Os seguintes dados da semana corrente e anterior são apresentados no painel: receita total, número total de agendamentos, número total de treinadores, número total de hora agendadas por treinadores.

\begin{figure}[H]
    \centering
    \includegraphics[width=0.4\textwidth]{pfc/figuras/gym-dashboard.png}
    \caption{Tela de dashboard da academia}
    \label{fig:gym-dashboard}
\end{figure}

\subsection{Agenda Semanal para Locação}
Ao acessar pela primeira vez a interface das academias, o usuário é direcionado para uma tela (ver Figura \ref{fig:gym-block-onboard}) onde são passadas informações de como funciona a agenda de blocos de horários semanais para locação da academia. A aplicação permite que as academias cadastrem blocos de horários fixos para cada dia da semana. Cada bloco contém a informação de taxa cobrada por hora, máximo número de pessoas permitidas e limites inferior e superior de horário para sessões. A partir destes blocos, os treinadores podem agendar sessões dentro dos mesmos (selecionando o número de clientes que eles vão levar, respeitando o limite máximo, e escolhendo o horário de início e término de uma sessão dentro dos limites de horário do bloco).

A Figura \ref{fig:gym-add-block} ilustra a tela de adição de um bloco semanal. As seguintes informações devem ser passadas: horários limites de início e término das sessões de treino, seleção de dias da semana nos quais o bloco será aplicado (ou seja, se academia quer disponibilizar blocos idênticos para diferentes dias da semana, ela utiliza esta opção), máximo número de pessoas permitidas dentro do bloco em um dado momento e, por último, taxa por hora cobrada dos treinadores.

Após a adição de blocos, os mesmos são exibidos na agenda (Figura \ref{fig:gym-block}) na posição correspondente aos limites de horário do bloco (blocos não podem ter limites de horários sobrepostos). Ao clicar no botão "Edit", o usuário pode editar as informações do bloco previamente fornecidas.

\begin{figure}[H]
	\centering
    \begin{subfigure}[b]{0.3\textwidth}
        \includegraphics[width=\textwidth]{pfc/figuras/gym-block-structure-onboard.png}
        \caption{Instruções da agenda semanal}
        \label{fig:gym-block-onboard}
    \end{subfigure}
    ~
	\begin{subfigure}[b]{0.3\textwidth}
        \includegraphics[width=\textwidth]{pfc/figuras/gym-add-block.png}
        \caption{Adição de novo bloco semanal}
        \label{fig:gym-add-block}
    \end{subfigure}
    ~
    \begin{subfigure}[b]{0.3\textwidth}
        \includegraphics[width=\textwidth]{pfc/figuras/gym-block-structure.png}
        \caption{Agenda semanal da academia populada}
        \label{fig:gym-block}
    \end{subfigure}
    ~
    \caption{Fluxo de criação de blocos de horários para agenda semanal das academias}
    \label{fig:}
\end{figure}

\subsection{Calendário de Sessões Agendadas}
A tela de calendário de sessões agendadas (Figura \ref{fig:gym-calendar}) mostra informações gerais de agendamento de sessões filtradas por data. Na parte superior da interface, é possível selecionar a data em que se deseja obter informações através de um calendário. Em seguida dados da data selecionada são exibidos (número de treinadores e clientes do dia e receita esperada para o dia). Por fim, uma agenda com os blocos do dia é apresentada, com a indicação de quantos treinadores e seus respectivos clientes marcaram sessões de treino em cada bloco.

Ao clicar o botão "View details" localizado em cada um dos blocos, a tela de detalhes de agendamentos (Figura \ref{fig:gym-booking-detail}) é exibida. Nesta tela, todas as sessões agendadas no bloco selecionado são listadas. Cada sessão é detalhada em um card, contendo o nome completo do treinador, horários de início e término da sessão, total a pagar e número de clientes que o mesmo vai levar ao estabelecimento.

\begin{figure}[H]
	\centering
    \begin{subfigure}[b]{0.4\textwidth}
        \includegraphics[width=\textwidth]{pfc/figuras/gym-booking-calendar.png}
        \caption{Calendário de sessões}
        \label{fig:gym-booking-calendar}
    \end{subfigure}
    ~
	\begin{subfigure}[b]{0.4\textwidth}
        \includegraphics[width=\textwidth]{pfc/figuras/gym-booking-detail.png}
        \caption{Detalhes da sessão}
        \label{fig:gym-booking-detail}
    \end{subfigure}
    ~
    \caption{Telas do calendário de sessões agendadas na academia}
    \label{fig:gym-calendar}
\end{figure}

\subsection{Perfil do Estabelecimento}
A tela de perfil da academia além de exibir todos os dados passados pelo usuário no momento do cadastro, também mostra um indicador da nota do estabelecimento (de $1$ a $5$, calculada como sendo a média das avaliações de sessões de treino) e um mapa com a marcação da localização da academia. A interface permite que o usuário clique no botão "Edit" para que possa realizar edições no perfil previamente preenchido no momento do cadastro.

\begin{figure}[H]
    \centering
    \includegraphics[width=0.4\textwidth]{pfc/figuras/gym-profile.png}
    \caption{Tela de perfil da academia}
    \label{fig:gym-profile}
\end{figure}

% *********************
% Interface treinadores
% *********************
\section{Interface para Treinadores}
Nesta secção, é apresentada a interface para usuários que se cadastram como treinadores. As principais funcionalidades implementadas para a versão piloto do aplicativo são abordadas: busca por academias, agendamento de sessões de treino, calendário de sessões agendadas e perfil de treinador.  

Na interface para treinadores, o usuário tem a opção de navegar por cinco telas principais, selecionadas a partir da barra de navegação (Figura \ref{fig:tr-tabbar}) localizada na região inferior do aplicativo. Da esquerda para a direita, as opções da barra de navegação são: mapa de busca, academias favoritadas, acesso rápido à academias recentes, calendário de sessões agendadas e perfil de treinador.

\begin{figure}[H]
    \centering
    \includegraphics{pfc/figuras/tr-tabbar.png}
    \caption{Barra de navegação da interface para treinadores}
    \label{fig:tr-tabbar}
\end{figure}

\subsection{Busca por Academias}
O aplicativo permite que treinadores realizem buscas por academias através de três fluxos diferentes. Todos os três caminhos direcionam o usuário para a tela de visualização de perfil da academia (Figura \ref{fig:tr-gym-profile-view}) para treinadores. A partir desta tela o usuário pode realizar o agendamento de sessões de treino, abordado na próxima sub-secção.

\begin{figure}[H]
    \centering
    \includegraphics[width=0.4\textwidth]{pfc/figuras/tr-gym-profile-view.png}
    \caption{Visão de treinador do perfil de academia}
    \label{fig:tr-gym-profile-view}
\end{figure}

O primeiro método de busca é feito a partir da tela do mapa de busca, ilustrado na Figura \ref{fig:tr-home-map}. O mapa indica a localização de academias, cadastradas na plataforma, com uma marcação preta (gota de suor). Dentro destas é exibida a taxa cobrada pela academia no próximo bloco disponível (informação que o back-end pode retornar pela API passando-se o horário em que o usuário realizou a busca). O mapa é renderizado dinamicamente através de integração feita com o serviço de geolocalização da Google, que será abordado no próximo capítulo. Na parte superior da tela, uma entrada de texto pode ser utilizada para realizar filtros de busca. Ao usuário digitar o nome ou endereço do estabelecimento, opções relacionadas à filtragem são apresentadas na tela (Figura \ref{fig:tr-map-search}). Em seguida, ao selecionar uma das opções, o mapa move-se até o local da academia e apresenta as informações básicas da mesma (Figura \ref{fig:tr-map-selected}).

\begin{figure}[H]
	\centering
    \begin{subfigure}[b]{0.3\textwidth}
        \includegraphics[width=\textwidth]{pfc/figuras/tr-map.png}
        \caption{Mapa de busca}
        \label{fig:tr-home-map}
    \end{subfigure}
    ~
	\begin{subfigure}[b]{0.3\textwidth}
        \includegraphics[width=\textwidth]{pfc/figuras/tr-map-search.png}
        \caption{Filtro de busca}
        \label{fig:tr-map-search}
    \end{subfigure}
    ~
    \begin{subfigure}[b]{0.3\textwidth}
        \includegraphics[width=\textwidth]{pfc/figuras/tr-home-map.png}
        \caption{Academia selecionada}
        \label{fig:tr-map-selected}
    \end{subfigure}
    ~
    \caption{Busca por academias através de mapa integrado}
    \label{fig:tr-gym-search}
\end{figure}

O segundo método de seleção de academia é feito através da tela de favoritos (Figura \ref{fig:tr-favorite}). Academias podem ser favoritadas através da tela de perfil da mesma (Figura \ref{fig:tr-gym-profile-view}) à marcação do símbolo de coração. Por fim, academias podem ser acessadas a partir da tela de academias recentes (Figura \ref{fig:tr-recent}). Academias são consideradas recentes quando quando uma sessão na mesma foi agendada. A tela apresenta as duas academias com sessões agendadas mais recentemente, além de também apresentar academias favoritadas no formato de lista.

\begin{figure}[H]
	\centering
    \begin{subfigure}[b]{0.4\textwidth}
        \includegraphics[width=\textwidth]{pfc/figuras/tr-favorite.png}
        \caption{Busca por favoritos}
        \label{fig:tr-favorite}
    \end{subfigure}
    ~
	\begin{subfigure}[b]{0.4\textwidth}
        \includegraphics[width=\textwidth]{pfc/figuras/tr-my-gyms.png}
        \caption{Busca por recentes}
        \label{fig:tr-recent}
    \end{subfigure}
    ~
    \caption{Busca por academias favoritas e recentes}
    \label{fig:tr-alternative-search}
\end{figure}

\subsection{Agendamento de Sessão}
Após o usuário acessar o perfil de uma academia, o mesmo pode dar início ao processo de agendamento de uma sessão de treino. Blocos disponíveis da academia são apresentados na parte inferior do perfil (Figura \ref{fig:tr-gym-profile-timetable}). Um calendário pode ser utilizado para selecionar a data desejada. Ao clicar em um dos blocos disponíveis, o usuário pode ver quantas vagas restam para determinada faixa de horário. Ao clicar o botão "Book now", o processo de agendamento é iniciado.

\begin{figure}[H]
    \centering
    \includegraphics[width=0.4\textwidth]{pfc/figuras/tr-gym-profile-2.png}
    \caption{Visão de treinador dos blocos disponíveis da academia}
    \label{fig:tr-gym-profile-timetable}
\end{figure}

O primeiro passo do agendamento de uma sessão de treino é ilustrado na Figura \ref{fig:tr-new-booking}. O usuário deve informar os seguintes dados da sessão: nome da sessão, horário de início, duração da sessão, número de clientes que levará (limitado a quatro). Prosseguindo, uma tela exibe o total a ser pago (Figura \ref{fig:tr-booking-payment}). Ao confirmar, uma transação é emitida ao serviço de pagamento (os dados de cartão de crédito do usuário não são utilizados na versão piloto do aplicativo, uma transação simulada é realizada). Por fim, o usuário visualiza todas as informações da sessão em uma tela de confirmação do agendamento (Figura \ref{fig:tr-booking-confirmed}). Nesta tela, o usuário pode tomar duas ações: fazer outro agendamento (opção "Make another booking"), ação que direciona o aplicativo para tela de academias recentes; ou enviar um SMS para seu cliente (opção "SMS client"), ação que gera uma mensagem de texto padrão e abre o aplicativo no dispositivo do usuário responsável por enviar SMS.

\begin{figure}[H]
	\centering
    \begin{subfigure}[b]{0.3\textwidth}
        \includegraphics[width=\textwidth]{pfc/figuras/tr-new-booking.png}
        \caption{Criação da sessão}
        \label{fig:tr-new-booking}
    \end{subfigure}
    ~
	\begin{subfigure}[b]{0.3\textwidth}
        \includegraphics[width=\textwidth]{pfc/figuras/tr-booking-payment.png}
        \caption{Pagamento}
        \label{fig:tr-booking-payment}
    \end{subfigure}
    ~
    \begin{subfigure}[b]{0.3\textwidth}
        \includegraphics[width=\textwidth]{pfc/figuras/tr-booking-confirmed.png}
        \caption{Confirmação}
        \label{fig:tr-booking-confirmed}
    \end{subfigure}
    ~
    \caption{Processo de agendamento de sessão de treino}
    \label{fig:tr-booking}
\end{figure}

\subsection{Calendário de Sessões Agendadas}
O usuário pode acessar as sessões agendadas pelo mesmo através da tela ilustrada na Figura \ref{fig:tr-calendar}. Nesta tela, um calendário na parte superior é apresentado, possibilitando a seleção de data. Abaixo, todas as sessões agendadas pelo treinador são exibidas. Cada sessão é representada por um card, contendo as informações passadas previamente no momento do agendamento. Através do card, o usuário pode tomar duas ações: visualizar o perfil da academia em que a sessão vai ocorrer ou cancelar a sessão.

\begin{figure}[H]
    \centering
    \includegraphics[width=0.4\textwidth]{pfc/figuras/tr-calendar.png}
    \caption{Calendário de sessões agendadas pelo treinador}
    \label{fig:tr-calendar}
\end{figure}

\subsection{Perfil do Treinador}
A tela de perfil do treinador (Figura \ref{fig:tr-profile-info}) exibe todos os dados, relacionados ao perfil, que o usuário cadastrou na plataforma. Ao clicar em "Edit", pode-se editar o perfil de treinador. Ao clicar no ícone localizado no canto superior direto da tela, o usuário é direcionado à tela de dashboard (Figura \ref{fig:tr-dashboard}). O dashboard do treinador contém as seguintes informações: gasto total com agendamentos, receita proveniente de clientes (dado não disponível na versão piloto do aplicativo), número total de agendamentos, clientes e horas de treino na semana corrente e passada.

\begin{figure}[H]
	\centering
    \begin{subfigure}[b]{0.4\textwidth}
        \includegraphics[width=\textwidth]{pfc/figuras/tr-profile.png}
        \caption{Informações gerais}
        \label{fig:tr-profile-info}
    \end{subfigure}
    ~
	\begin{subfigure}[b]{0.4\textwidth}
        \includegraphics[width=\textwidth]{pfc/figuras/tr-dashboard.png}
        \caption{Dashboard}
        \label{fig:tr-dashboard}
    \end{subfigure}
    ~
    \caption{Perfil do treinador}
    \label{fig:tr-profile}
\end{figure}

\chapter{Desenvolvimento e Testes do Aplicativo}

\section{Tecnologias Utilizadas}

\section{Arquitetura Geral da Aplicação}

\section{Desenvolvimento da Interface Gráfica}

\section{Integração de Sistemas}

\subsection{Back-end}

\subsection{Serviços Web}

\subsubsection{Geolocalização}

\subsubsection{Pagamento}

\subsubsection{Envio de SMS}

\section{Testes Automatizados de Interface Gráfica}

\subsection{Implementação dos Testes}

\subsubsection{EarlGrey}

\subsubsection{Appium}

\subsubsection{XCUITest}
\chapter{Implementação e Testes do Aplicativo} \label{cap:development}
Neste capítulo, são apresentados os detalhes de implementação e testes do aplicativo. 

% **********************
% Integração de Sistemas
% **********************
\section{Integração de Sistemas}
O desenvolvimento da aplicação evolveu a integração de múltiplos sistemas de software, tanto internos quanto externos. Nesta secção, primeiro é apresentada a integração com o sistema de back-end, desenvolvido internamente. Em seguida, o mesmo é feito para os sistemas externos. Por último, a ferramenta utilizada para testes das integrações feitas por RESTful API's são apresentadas. 

\subsection{Back-end}
A integração com o back-end foi feita por meio de uma API REST. A API fornece endpoints que retornam informações relacionadas à lógica de negócio da aplicação e permitem o cadastro de dados do usuário.

O formato utilizado para troca de dados foi o JSON \todo{referenciar fig json}. Este formato foi utilizado por apresentar uma notação simples para entendimento e pelo fato de ser utilizado em larga escala em integrações de serviço web. Do ponto de vista humano, a notação é de fácil visualização por não apresentar uma sintaxe verbosa. Além disso, o formato tem suporte nativo de diversas linguagens de programação, incluindo o Swift, utilizado para realizar a implementação.

\missingfigure{exemplo de json}

A API contém algumas características relacionadas à segurança do sistema. A API fornece diferentes níveis de acesso para as chamadas a fim de proteger dados sensíveis. Todo usuário recebe um \textit{token} no momento em que realiza o login ou é registrado no aplicativo. Este \textit{token} é utilizado para realizar a autenticação de chamadas da API, permitindo a identificação do usuário que realizou a requisição pelo aplicativo. Através da identificação, o back-end pode retornar ou negar o pedido de chamada de API de acordo com o nível de acesso do usuário. Além da autenticação e autorização de usuários no sistema, existem dois ambientes de uso da API: um para fase de desenvolvimento e testes, outro para o aplicativo em produção. Ambos os ambientes contêm o mesmo conjunto de chamadas de API disponíveis, distinguindo-se apenas no uso de diferentes bancos de dados (desenvolvimento e produção). Este recurso permite a separação do uso do aplicativo para fins de testes, realizados pelo time de desenvolvimento e pelo cliente, e para fins de uso real, feito pelo usuário final. A separação dos ambientes é feita a partir da diferenciação da URL base da API REST.

No código do aplicativo, as chamadas de API são feitas com o auxílio de uma biblioteca para a linguagem Swift. A biblioteca, denominada Alamofire \todo{ref alamofire}, é uma camada de abstração para a utilização de recursos de rede do sistema iOS.

O recebimento dos dados da API é feita a partir de uma tradução automática do formato JSON para objetos do Swift. Através do recurso denominado \textit{Codable} da linguagem e da configuração de parâmetros de transformação de dados. Primeiramente, são escolhidos os agentes de codificação e decodificação para atender o formato JSON. Em seguida é utilizado o parâmetro de configuração que transforma a notação \textit{snake case} da API para a notação \textit{camel case}, utilizada no código \todo{ref fig snake camel}.

\subsection{Serviços Externos}
O desenvolvimento do aplicativo teve como requisito a implementação dos sistemas de geolocalização, pagamento e envio de SMS. No projeto, optou-se pela utilização destes sistemas na forma de serviços devido ao grande esforço envolvido na implementação e manutenção dos mesmos. Existem diversas alternativas consolidadas há anos no mercado para cada um dos sistemas citados. As soluções escolhidas e as integrações realizadas são apresentadas a seguir. 

\subsubsection{Geolocalização}
O aplicativo apresentou como um de seus requisitos funcionais a presença de um sistema de mapa, assim como a utilização da localização do usuário. Foi escolhido o serviço de geolocalização do Google Maps \todo{referenciar google maps api}, pelo motivo do mesmo disponibilizar uma SDK para o sistema iOS de fácil utilização, com boa documentação e suporte em comunidades da internet. Para detectar a localização do usuário, utilizou-se o recurso de GPS nativo do sistema iOS, disponibilizado através do framework \textit{Core Location}, parte da camada de \textit{Core Services} do sistema operacional.

O serviço de geolocalização do Google Maps é utilizado em diversas partes do aplicativo. No momento do cadastro de uma academia, o serviço é utilizado para o preenchimento do campo endereço (Figura \ref{fig:address-field}), obrigatório para todos os estabelecimentos cadastrados na plataforma. Requisições são feitas para a API do Google Maps à medida que o usuário acrescenta caracteres no campo de endereço. Como retorno, opções de endereços semelhantes ao digitado são apresentadas na interface (Figura \ref{fig:search-delfino}). Uma vez que o usuário escolhe um dos endereços disponíveis na busca, o serviço de geolocalização retorna as coordenadas referentes ao endereço. Estas coordenadas são enviadas ao back-end para armazenamento no banco de dados, junto aos demais dados da academia, e posteriormente são utilizadas para demarcação dos estabelecimentos no mapa.

\begin{figure}[H]
	\centering
    \begin{subfigure}[b]{0.4\textwidth}
        \includegraphics[width=\textwidth]{pfc/figuras/delfino-conti-2.png}
        \caption{Preenchimento do campo de endereço}
        \label{fig:address-field}
    \end{subfigure}
    ~
	\begin{subfigure}[b]{0.4\textwidth}
        \includegraphics[width=\textwidth]{pfc/figuras/delfino-conti.png}
        \caption{Busca por endereços através da API do Google Maps}
        \label{fig:search-delfino}
    \end{subfigure}
    ~
    \caption{Etapa de preenchimento do campo endereço no cadastro de uma academia}
    \label{fig:addreess-field-fill}
\end{figure}

Outra utilização do serviço de geolocalização é o sistema de mapas, recorrente em diversas telas do aplicativo. O sistema de mapas é utilizado de maneira estática (sem a interação do usuário) nas telas de boas vindas (Figura \ref{fig:gym-welcome}), de perfil (Figuras \ref{fig:gym-profile} e \ref{fig:tr-gym-profile-view}) e de cadastro (Figura \ref{fig:address-field}) das academias e, também, na tela de confirmação de agendamento de treino (Figura \ref{fig:tr-booking-confirmed}). Na interface do treinador (Figura \ref{fig:tr-gym-search}), o mapa é renderizado na tela dinamicamente conforme a interação do usuário. O sistema de mapas permite a utilização de movimentos de pinça para alterar o zoom, além da movimentação simples do mapa através de toques na tela. Sempre que um mapa é renderizado na tela do aplicativo, uma requisição é feita ao back-end a fim de identificar as academias que estão presentes naquela região do mapa. O ponto central da região do mapa é enviado junto com um raio de busca (foi utilizado como padrão um raio de busca de $5\,km$) para o back-end. Então, uma lista com todas as academias da região é retornada e, a partir dos dados de coordenadas salvos, marcações são feitas no mapa.
                               
O framework \textit{Core Location} é utilizado em conjunto com o sistema de mapas para detectar a posição atual do usuário. Para utilizar o recurso nativo do sistema iOS, uma requisição de permissão de uso do sistema de localização do dispositivo deve ser feita ao usuário (Figura \ref{fig:gps-permission}). Em seguida, uma marcação especial é feita no mapa (ícone de músculos do braço - Figura \ref{fig:muscle-marker}) para indicar a localização corrente. A partir deste instante, um observador é configurado para que o sistema operacional emita um aviso ao aplicativo quando o usuário se movimentar. A configuração é feita com um parâmetro de precisão de $10/,m$. Assim, quando o usuário se desloca mais que esta distância, um aviso é emitido ao aplicativo e, em seguida, a posição do marcador no mapa é atualizada.

\begin{figure}[H]
	\centering
    \begin{subfigure}[b]{0.38\textwidth}
        \includegraphics[width=\textwidth]{pfc/figuras/gps-permission.png}
        \caption{Permissão de uso do GPS}
        \label{fig:gps-permission}
    \end{subfigure}
    ~
	\begin{subfigure}[b]{0.4\textwidth}
        \includegraphics[width=\textwidth]{pfc/figuras/tr-home.png}
        \caption{Marcador de posição}
        \label{fig:muscle-marker}
    \end{subfigure}
    ~
    \caption{Marcação da posição atual do usuário no mapa}
    \label{fig:user-position}
\end{figure}

\subsubsection{Pagamento}
Um sistema de pagamentos foi utilizado para que as transações financeiras possam ocorrer dentro da plataforma. O serviço escolhido para implementação desta funcionalidade foi o Stripe \todo{referenciar stripe}. O serviço foi escolhido por apresentar a melhor usabilidade do ponto de vista de desenvolvimento, apresentando uma boa documentação e estrutura de API's e disponibilização de SDK para iOS com diversos recursos embutidos, como validação de campos de cartão de crédito.

O funcionamento do sistema de pagamentos da aplicação é ilustrado na Figura \ref{fig:payment-system}. Por questões de segurança, nenhum dado sensível (como número do cartão de crédito ou conta bancária) é armazenado no banco de dados da aplicação. Esta tarefa é delegada ao Stripe, assim como a realização de todas as transações financeiras. Na plataforma do Stripe, são criadas contas virtuais para cada academia, assim como uma conta virtual do Gyymi. Além disso, um processo de \textit{tokenização} é feito para a representação dos cartões de crédito dos treinadores. A \textit{tokenização} funciona da seguinte forma (ver Figura \ref{fig:seq-diagram-token}): o usuário informa os dados do cartão através da interface gráfica do aplicativo iOS; os dados são enviados para o Stripe, onde são armazenados; um \textit{token} de identificação única para representar o cartão de crédito é gerado e enviado ao aplicativo iOS; o \textit{token} é enviado ao back-end, que após validar o mesmo com o Stripe, armazena-o no banco de dados da aplicação; por último, o back-end retorna uma resposta ao aplicativo iOS informando que o processo de \textit{tokenização} do cartão de crédito foi concluído com sucesso. Dessa forma, a aplicação pode identificar cartões de crédito cadastrados e, em sequência, emitir pedidos de transação com o uso do \textit{token}.

Existem três tipos de transações dentro do Stripe:
\begin{itemize}
    \item Cobrança: quando um pagamento é feito pelo treinador, o Stripe emite uma cobrança no cartão de crédito cadastrado.
    \item Transferência: diz respeito as transferências entre contas virtuais do Stripe. Todos os pagamentos emitidos são enviados primeiramente à conta virtual do Gyymi, que retira uma taxa do pagamento como forma de cobrança pelo serviço de intermédio e, então, repassa o restante do valor para as contas virtuais das academias.
    \item Pagamento: ao final de um intervalo de tempo pré-definido, o dinheiro das contas virtuais é repassado para as contas bancárias reais.
\end{itemize}

\begin{figure}[H]
    \centering
    \includegraphics[width=0.8\textwidth]{pfc/figuras/payment-system-stripe.png}
    \caption{Funcionamento do sistema de pagamentos}
    \label{fig:payment-system}
\end{figure}

\begin{figure}[H]
    \centering
    \includegraphics[width=0.8\textwidth]{pfc/figuras/seq-diagram-token.png}
    \caption{Diagrama de sequência da dinâmica de \textit{tokenização} do cartão de crédito}
    \label{fig:seq-diagram-token}
\end{figure}

A implementação dos esquemas apresentados não foi concluída para a fase piloto do aplicativo. Para o piloto, foi utilizado um ambiente fictício que o Stripe fornece para testes. Transações fictícias foram criadas no Stripe ao final dos agendamentos de treinos, emitindo-se cobranças em cartões de crédito fictícios.

\subsubsection{Envio de SMS}
O sistema de envio de SMS foi utilizado para cumprir o requisito funcional de envio de SMS para verificação de número telefônico. A verificação é feita somente no cadsatro de novos treinadores (Figura \ref{fig:register-trainer-verification}). O serviço utilizado foi o Twilio \todo{citar twilio}.

A dinâmica de envio de SMS é ilustrada no diagrama de sequência da Figura \ref{fig:seq-diagram-sms}. O sistema opera da seguinte forma: o usuário preenche o cadastro através da interface gráfica do aplicativo iOS, em especial o campo de número do celular; os dados de cadastro são enviados e validados no back-end, que faz uma chamada de API para o Twilio com um pedido de envio de SMS para o número de celular cadastrado; o usuário, então, preenche o campo de código de verificação no aplicativo com o código recebido via SMS; o código de verificação é enviado ao back-end, que realiza a validação do mesmo com o Twilio; por fim, uma confirmação da validação do código de verificação é enviada ao aplicativo iOS, permitindo que o usuário prossiga com o cadastro.

\begin{figure}[H]
    \centering
    \includegraphics[width=0.8\textwidth]{pfc/figuras/seq-diagram-sms.png}
    \caption{Diagrama de sequência da dinâmica de envio de SMS para verificação de número telefônico}
    \label{fig:seq-diagram-sms}
\end{figure}

\subsection{Testes de RESTful API's}
Todas as integrações de sistemas apresentadas anteriormente foram realizadas utilizando RESTful API's. Com o intuito de agilizar e automatizar o processo de teste de chamadas de API, utilizou-se durante o projeto a ferramenta de software denominada Postman \todo{citar postman}. A ferramenta permite que conjuntos de chamadas de API's (coleções) sejam criadas, permitindo o teste das chamadas individualmente ou em conjunto (com o sequenciamento das chamadas).

A Figura \ref{fig:postman} ilustra a interface do Postman. Na esquerda, encontram-se as coleções, com as respectivas listas de chamadas de API. Na parte central superior, existem abas com as chamadas e seus parâmetros de configurações. Na parte central inferior, encontra-se um console onde a resposta da chamada é exibida.

\begin{figure}[H]
    \centering
    \includegraphics[eidth=0.8\textwidth]{pfc/figuras/postman.png}
    \caption{Postman - ferramente utilizada para testes de RESTful API's}
    \label{fig:postman}
\end{figure}

% *****************
% Interface Gráfica
% *****************
\section{Interface Gráfica}

criterio de testes e avaliacao

componentes reusaveis
botao
calendario
stepper
navbar

tratamento de erros
cadastro
regex
api

bibliotecas
image crop/resize
googlemapas


desafios/detalhes de implementação
espaço entre campos e botao
tamanhos de dispositivo
tabelas de horario
redirecionar insta/website
bolinha calendario

% ********************
% Testes Automatizados
% ********************
\section{Testes Automatizados de Interface Gráfica}

\subsection{Implementação dos Testes}

\subsubsection{EarlGrey}

\subsubsection{Appium}

\subsubsection{XCUITest}

\chapter{Resultados} \label{cap:results}

\section{Versão Piloto do Aplicativo}

\section{Testes Automatizados}

\section{Resultados para a Empresa}

% ----------------------------------------------------------
% Finaliza a parte no bookmark do PDF
% para que se inicie o bookmark na raiz
% e adiciona espaço de parte no Sumário
% ----------------------------------------------------------
\phantompart

% ----------------------------------------------------------
% ELEMENTOS PÓS-TEXTUAIS
% ----------------------------------------------------------
\postextual
\bibliography{referencias}
% \glossary
% \include{apendices}
% \include{anexos}

%---------------------------------------------------------------------
% INDICE REMISSIVO
%---------------------------------------------------------------------
\phantompart
\printindex

\end{document}
