\chapter{Contextualização}

\section{A Empresa}
\missingfigure{logo da Jungle}
A Jungle Devs \todo{referenciar logo-jungledevs} é uma empresa do setor de TI fundada em 2018 por ex-alunos do curso de Engenharia de Controle e Automação da UFSC, que trabalhavam juntos desde 2010. Sediada em Florianópolis, a empresa atua na área de desenvolvimento de software, com especialidade em produtos web e \textit{mobile}, e oferece programas de capacitação para futuros profissionais do setor tecnológico.

A empresa foi oficialmente fundada em fevereiro de 2018 por três engenheiros. Antonio, Andrio e Vinicius moravam e trabalhavam juntos em outra empresa do setor de TI, desenvolvendo produtos de tecnologia. Os três decidiram fundar a própria empresa com o conhecimento técnico adquirido pelos membros durante dois anos de trabalho na antiga empresa, somado aos mais de dez anos de experiência de Antonio na indústria de TI, que resultaram em forte rede de conexões com profissionais ao redor do mundo, em especial na África do Sul, Polônia e Vale do Silício, na Califórnia - Estados Unidos. A ideia de abrir o próprio negócio surgiu da identificação da necessidade por novos profissionais que o setor de tecnologia apresenta. Tendo em vista a possibilidade de capacitar pessoas fornecendo experiência prática com os projetos emergentes das parcerias profissionais, os três engenheiros decidiram criar a empresa com o foco no desenvolvimento de pessoas.

A partir da rede de conexões ao redor do mundo, empresa teve início com o desenvolvimento de projetos de aplicações \textit{web} e \textit{mobile} para clientes de diversos países, como Estados Unidos e Austrália. A proposta de capacitar novos profissionais de tecnologia pôde ser validada com outros três engenheiros, agora também sócios da Jungle Devs. Henrique, Pedro e Rodrigo ingressaram na empresa sem nenhuma experiência profissional prévia no desenvolvimento de produtos de tecnologia, e a partir do envolvimento dos três diretamente com o desenvolvimento de novas aplicações, formou-se um time. Com o feedback positivo dos parceiros e, também dos clientes, novos projetos começaram a surgir e, também, uma nova demanda por profissionais capacitados.

A Jungle Devs, então, surgiu com a missão de repensar a forma como pessoas aprendem e criam tecnologia, dando o início desse processo com um programa de capacitação denominado \textit{Academy}. Os participantes do \textit{Academy} passam por diferentes períodos de treinamento, partindo do aprendizado de conceitos básicos de programação até a atuação no desenvolvimento de um projeto real e conexão do novo profissional com o mercado de trabalho. Este trabalho de PFC está inserido na última etapa do programa com foco em desenvolvimento \textit{mobile}.

Em paralelo ao propósito de fomentar o mercado de trabalho do setor de tecnologia, a Jungle Devs oferece à empresas os serviços de idealização de produtos, design, desenvolvimento, revisão e testes, lançamento de produtos no mercado e serviços de manutenção. Assim, estes serviços são utilizados como plataforma de aprendizado para novos profissionais, tendo o suporte de um time de engenharia próprio da empresa.

Atualmente a empresa tem crescido organicamente a partir de indicações, tanto em número de pessoas como de projetos. De fevereiro à novembro de 2018, a empresa teve crescimento do número de pessoas de 7 para 13, somando participantes do \textit{Academy} e time de engenharia. Além disso, a empresa conta com uma rede de profissionais denominada \textit{Jungle Community} que atuam como freelancers nos projetos da empresa. Os projetos são em maioria desenvolvidos para clientes do exterior, contemplando países da América, África, Europa, Ásia e Oceania. O surgimento de novos projetos tem sido recorrente com a recomendação de novos clientes a partir dos atuais e anteriores, sendo este um bom indicador de qualidade da entrega dos produtos.

\section{A Aplicação}
A partir de uma recomendação de um cliente de outro projeto de software desenvolvido na Austrália, a Jungle Devs foi contratada para desenvolver um aplicativo. A aplicação, denominada Gyymi, tem o intuito de aproximar treinadores pessoais profissionais com academias de musculação. O cliente australiano identificou uma oportunidade de mercado com a ideia de um produto que pudesse tirar proveito da ociosidade de muitos estabelecimentos com infraestrutura para treinamentos de musculação, tanto de academias que já oferecem este serviço, mas também de locais que potencialmente poderiam oferecê-lo (como hotéis, condomínios, entre outros), principalmente os localizados em países desenvolvidos, como é o caso da Austrália.

O uso da tecnologia móvel de forma semelhante ao proposto pelo aplicativo - através de uma abordagem de economia colaborativa (\textit{sharing economy} em inglês) - já obteve sucesso em outros campos de atuação, como no setor de transportes e no ramo imobiliário \todo{citar sharing-economy}. O modelo de estratégia proposto permite que indivíduos e grupos se beneficiem com o uso de recursos subutilizados, oferecendo os mesmos na forma de serviço. Da mesma maneira, o Gyymi tem o potencial de oferecer a oportunidade de estabelecimentos voltados à atividades físicas otimizarem o uso de seus espaços e recursos, tendo em vista a demanda crescente da população mundial por exercícios físicos \todo{provar essa bagaça}.

O desenvolvimento do produto ocorrerá em diversas etapas. Inicialmente foi lançada uma versão piloto do aplicativo apenas no sistema \textit{iOS}, discutida dentro deste trabalho. Futuramente, uma versão para a plataforma \textit{Android} será lançada, aplicativo que também já está em fase de desenvolvimento. A versão piloto está em atual fase de testes com parceiros de estabelecimentos localizados em Sydney, na Austrália. Para o ano seguinte, está previsto o lançamento da primeira versão oficial do aplicativo.
