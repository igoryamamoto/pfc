\chapter{Conclusões e Perspectivas} \label{cap:conclusions}
O primeiro objetivo deste trabalho foi o desenvolvimento do aplicativo Gyymi para o sistema iOS. Através da execução do projeto utilizando abordagens de desenvolvimento ágil de software, a aplicação evolui de um estado apenas conceitual até uma versão testável durante o período do PFC. O aplicativo foi desenvolvido em cima de uma arquitetura de software envolvendo integrações de vários sistemas e serviços, que permitiram a realização de funcionalidades essenciais, como a localização por mapa e o sistema de pagamentos. Diferentes tecnologias e linguagens de programação foram utilizadas ao longo do projeto, optando-se por opções modernas e de código aberto, como a linguagem Swift. Todas essas características permitiram a construção com sucesso de um aplicativo de relativa complexidade em curto período de tempo.

A versão atual do aplicativo encontra-se na fase piloto. O sistema é capaz de cadastrar usuários e estabelecimentos em uma plataforma online. Administradores de academias podem registrar seu perfil usuário e cadastras as academias, fornecendo características do local. Enquanto treinadores podem cadastrar seu perfil profissional. Além do cadastro, a plataforma permite o agendamento de sessões de treino. Os agendamentos realizam o principal papel do aplicativo, que consiste em conectar e intermediar o relacionamento entre academias e treinadores. Academias podem estruturar suas agendas de disponibilidade e treinadores podem buscar por estabelecimentos e marcar horários.

As funcionalidades desenvolvidas para a versão piloto permitiram o cliente na Austrália desse início a fase de teste com seus parceiros. O aplicativo já está sendo testado por mais de vinte usuários entre administradores de academias e treinadores, com o intuito de coletar feedback e realizar melhorias para lançamentos oficiais em breve. Um novo ciclo de desenvolvimento visando o lançamento foi iniciado logo após o lançamento da versão piloto e está atualmente em curso. A perspectiva para a primeira versão do aplicativo é de ter a funcionalidade de pagamentos concluída, permitindo testes com cartões de crédito e contas bancárias reais. Assim, a aplicação pode iniciar o seu processo de monetização.

As contribuições do autor para o desenvolvimento do projeto foram de suma importância para o sucesso do aplicativo. As contribuições incluem parte majoritária da implementação do aplicativo para o sistema iOS e integrações de sistemas terceiros. Além disso, ao final do lançamento da versão piloto, um estudo de implementação de testes automatizados foi conduzido.

O estudo constituiu-se no segundo objetivo geral deste trabalho, visando o aprimoramento de processos relacionados a testes dentro da empresa Jungle Devs. Pesquisas de ferramentas e conceitos de testes automatizados foram feitas e implementações de testes relacionados a interface gráfica foram iniciados. Os testes foram feitos de modo a não alterar o modelo de trabalho corrente na empresa durante o projeto, porém visando buscar alternativas de melhorias futuras.

Dois casos de teste foram planejados, construídos e executados. Estes foram baseados na parte do Gyymi de cadastro, funcionalidade comum em muitos aplicativos e que, portanto, apresentou cenários relevantes para aplicações futuras. Os casos de teste permitiram a validação e comparação de uso de diferentes ferramentas de automação. Além disso, o estudo detectou oportunidades de melhoria dentro da empresa.

Por fim, este trabalho de PFC mostrou-se de extrema importância para a formação profissional e pessoal do autor. Diversos momentos de aprendizado foram proporcionados ao longo do programa de capacitação oferecido pela empresa, com mentorias e participação ativa em projeto com time experiente de profissionais.
