\chapter{Conceitos Básicos}
Este capítulo tem como objetivo apresentar alguns conceitos utilizados como base para a realização do presente trabalho. Algumas tecnologias serão expostas para facilitar o entendimento dos próximos capítulos. Também é discutido onde foi feito a utilização destas ferramentas e teorias no decorrer do trabalho, bem como as dificuldades encontradas.

\section{Métodos Ágeis de Desenvolvimento de Software}

\subsection{Agile}
\textit{Agile} (desenvolvimento ágil de software ou métodos ágeis) é um termo que se refere a abordagem de desenvolvimento de software em que requisitos e soluções evoluem através de um esforço colaborativo de times auto-geridos e multi-funcionais, de sues clientes e usuários finais. O conceito prega planejamentos adaptativos, desenvolvimento iterativo com melhorias contínuas e encoraja a rápida e flexível resposta a mudanças. O conceito é frequentemente posto em prática com o uso de frameworks de desenvolvimento, como \textit{Scrum} e \textit{Kanban}. O termo foi popularizado através do manifesto ágil de desenvolvimento de software \todo{citar agile-manifest}, que tem como valores:

\begin{itemize}
    \item Indivíduos e iterações mais que processos e ferramentas
    \item Software funcional mais que documentação abrangente
    \item Colaboração do cliente mais que negociação de contratos
    \item Responder a mudanças mais que seguir um plano
\end{itemize}

\subsection{Scrum}
\missingfigure{figura do scrum.org}

\textit{Scrum} é um framework com o propósito de abordar problemas adaptativos complexos, entregando produtos com o maior valor agregado possível de forma produtiva e criativa. O \textit{Scrum} tem sido usado para o gerenciamento de produtos desde o início dos anos 90. Ele não é um processo, técnica ou método definitivo, mas sim uma estrutura para empregar vários processos e técnicas. O \textit{Scrum} evidencia a eficácia relativa do gerenciamento de produtos e técnicas de trabalho para que o produto, a equipe e o ambiente de trabalho possam melhorar continuamente. O framework fundamenta-se nos conceitos de transparência, inspeção e adaptação e tem como valores: compromisso, coragem, foco, abertura e respeito.

O framework do \textit{Scrum} consiste em times com atribuições associadas, eventos e artefatos. Estes elementos são elencados a seguir:
\begin{itemize}
    \item Atribuições nos times:
    \begin{itemize}
        \item \textit{Product Owner}: responsável por maximizar o valor do produto resultante do trabalho do time de desenvolvimento. É uma única pessoal responsável por gerenciar o backlog do produto (\textit{Product Backlog}).
        \item \textit{Development Team}: é o time de desenvolvimento. Consiste de um time de profissionais que realizam o trabalho de entregar um versão pronta de um incremento do produto. Os times de desenvolvimento devem ter seus respectivos trabalhos auto-geridos de tal forma a aumentar a eficiência e eficâcia do projeto.
        \item \textit{Scrum Master}: responsável por promover e dar apoio as regras do \textit{Scrum} dentro do time. Esta pessoa deve ajudar os demais a compreender a teoria por trás do framework, práticas, regras e valores.
    \end{itemize}
    \item Eventos:
    \begin{itemize}
        \item \textit{Sprint}: é o evento central do \textit{Scrum}, consistindo de uma janela limitada de tempo com duração de um mês ou menos. Durante este período temporal, a equipe deve concluir um versão incremental do produto a ser entregue. As durações das \textit{sprints} devem ser consistentes ao longo do projeto, com uma \textit{sprint} iniciando-se imediatamente ao término da anterior.
        \item \textit{Sprint Planning}: o trabalho a ser realizado durante a \textit{sprint} é planejado durante a sessão de \textit{sprint planning} através do trabalho colaborativo de todos os membros do time. A duração deste evento deve ser limitado proporcionalmente ao tamanho da janela de tempo da \textit{sprint}, sendo $8h$ o limite para uma sprint de um mês
        \item \textit{Daily Scrum}: é um evento diário executado pelo time de desenvolvimento, limitado a quinze minutos. Ocorre todos os dias no mesmo horário com o objetivo de inspecionar o trabalho executado no dia anterior e anteceder discussões sobre o trabalho futuro.
        \item \textit{Sprint Review}: evento que ocorre ao final da \textit{sprint} com o intuito de avaliar o progresso do time em relação ao incremento do produto. Durante a sessão, limita a $4h$ para uma \textit{sprint} de um mês, o time e \textit{stakeholders} discutem o que foi feito e proveem feedback.
        \item \textit{Sprint Retrospective}: evento que ocorre após a \textit{sprint review} e antecede a \textit{sprint planning}, limitado a $3h$ para uma \textit{sprint} de um mês. Durante o evento, o time tem a oportunidade de avaliar seu próprio desempenho e criar um plano de melhorias a serem alcançados na próxima \textit{sprint}.
    \end{itemize}
    \item Artefatos:
    \begin{itemize}
        \item \textit{Product Backlog}: consiste em uma lista ordenada de todas as tarefas a serem executadas que são de conhecimento até o momento. É uma única fonte de requisitos para qualquer mudança a ser implementada no produto. Ele nunca está completo, mudando constantemente conforme a evolução do produto. Ele lista todas as funcionalidades, requisitos, melhorias e consertos que devem ser aplicados ao produto, bem como uma descrição do que é considerada uma tarefa completa.
        \item \textit{Sprint Backlog}: é um conjunto de itens selecionados do \textit{backlog} do produto com um plano do incremento do produto que será entregue ao final da \textit{sprint}.
    \end{itemize}
\end{itemize}

O projeto do Gyymi seguiu a seguinte estrutura de \textit{Scrum}: o \textit{product owner} e o \textit{scrum master} foram uma única pessoa, devido ao tamanho reduzido da equipe, não foi julgado necessário duas pessoas para tal atribuições. O time de desenvolvimento foi composto por um desenvolvedor back-end, um líder técnico \textit{iOS}, um desenvolvedor \textit{iOS} (função exercida pelo aluno deste PFC), um desenvolvedor \textit{Android} e um designer. A \textit{sprint} teve durações de duas semanas, o \textit{daily scrum} foi executado todos os dias com duração máxima de quinze minutos e os eventos de \textit{sprint review}, \textit{sprint retrospective} e \textit{sprint planning} foram todos executados durante a mesma sessão para otimizar o tempo, visto que o projeto como versão piloto não necessitava de sessões específicas para cada evento.

\section{Desenvolvimento de Aplicativos iOS}

\subsection{Sistema Operacional e Arquitetura da Plataforma}

\subsection{Ambiente de Desenvolvimento}

\subsection{Swift}

\subsection{Gerenciamento de Dependências}

\section{Versionamento de Código}

\section{RESTful API's}

\section{Testes Automatizados}
\missingfigure{Figuras classicas dos testes, modelo em V, piramide etc}
\subsection{Testes de UI}
