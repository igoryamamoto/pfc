\chapter{Introdução}

\section{Motivação}

A demanda por soluções tecnológicas na área da saúde vem crescendo a nível global, com os  \textit{smartphones} ganhando destaque entre as ferramentas utilizadas \todo{citar mhealth-who}. O termo \textit{mHealth} (do inglês, \textit{mobile health}) é utilizado para o estudo da prática da medicina e saúde pública através do uso da tecnologia móvel (\textit{mobile} em inglês). Diversos tipos de segmentos da área da saúde, como assistência médica, nutrição, gestão de estilo de vida e bem-estar compõem um mercado de aplicações de \textit{mHealth} que já ultrapassa o valor de mercado de $25$ bilhões de dólares anuais, com mais de $325.000$ aplicativos disponíveis, somando um total de $3.6$ bilhões de downloads nas lojas virtuais (estimativa de 2017).\todo{citar mhealth-economics}

O segmento da área da saúde que mais se destaca no uso de tecnologias móveis, obtendo o maior número de assinaturas de usuários e receita total\todo{citar mhealth-market}, é o de atividades físicas (\textit{fitness} em inglês). Tendo em vista este fator, a Jungle Devs, empresa onde foi realizado este trabalho de PFC, foi contratada por um cliente da Austrália para desenvolver um aplicativo móvel envolvendo problemas dentro deste segmento. A ideia do projeto é desenvolver aplicativos para as plataformas \textit{iOS} e \textit{Android} que facilitem a conexão entre academias de musculação e treinadores pessoais profissionais, permitindo que aquelas sejam locadas por estes.

O uso da tecnologia móvel de forma semelhante ao proposto - através de uma abordagem de economia colaborativa (\textit{sharing economy} em inglês) - já obteve sucesso em outros campos de atuação \todo{citar sharing-economy}. Exemplos como \textit{Airbnb}, no ramo imobiliário, e \textit{Uber}, no setor de transportes, mostram como indivíduos e grupos podem se beneficiar com o uso de recursos subutilizados, oferecendo os mesmos na forma de serviço. Oportunidades para usufruir deste benefício da economia colaborativa foram detectadas na proposta do aplicativo estudado neste PFC, a exemplo de academias e locais com infraestrutura disponível para musculação (hotéis, condomínios, entre outros) que poderiam otimizar o uso de seus espaços e recursos, tendo em vista a demanda existente da população por exercícios físico.

Em face ao cenário apresentado, a construção do aplicativo adotou a abordagem de desenvolvimento ágil (\textit{agile} em inglês), com o uso de técnicas como o \textit{Scrum} para assegurar a rápida prototipação da solução, permitindo a iteração de desenvolvimento frequente baseada no \textit{feedback} do cliente de forma que o aplicativo atenda as necessidades reais dos usuários. Além de selecionar e fazer uso de ferramentas de programação modernas 

\section{Objetivos}-

- Desenvolvimento do aplicativo iOS
    - Interface gráfica
    - Integração com sistemas
    
- Estudo, pesquisa e implementação de testes automatizados de interface 
    - Pesquisa de ferramentas
    - Experimentos
    - Estudo de viabilidade

\section{Estrutura do Documento}

\todo{explicar estrutura do doc}
Este trabalho está dividido da seguinte forma: 

No capítulo 2, a contextualização

No capítulo 3, os conceitos básicos

No capítulo 4, a descrição conceitual

No capítulo 5, o desenvolvimento

No capítulo 6, os resultados

E por fim, no capítulo 7, as conclusoes 
