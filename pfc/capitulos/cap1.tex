\chapter{Introdução}
% *********
% Motivação
% *********
\section{Motivação}
A demanda por soluções tecnológicas na área da saúde vem crescendo a nível global, com os  \textit{smartphones} ganhando destaque entre as ferramentas utilizadas \todo{citar mhealth-who}. O termo \textit{mHealth} (do inglês, \textit{mobile health}) é utilizado para o estudo da prática da medicina e saúde pública através do uso da tecnologia móvel (\textit{mobile} em inglês). Diversos tipos de segmentos da área da saúde, como assistência médica, nutrição, gestão de estilo de vida e bem-estar compõem um mercado de aplicações de \textit{mHealth} que já ultrapassa o valor de mercado de $25$ bilhões de dólares anuais, com mais de $325.000$ aplicativos disponíveis, somando um total de $3.6$ bilhões de downloads nas lojas virtuais (estimativa de 2017).\todo{citar mhealth-economics}

O segmento da área da saúde que mais se destaca no uso de tecnologias móveis, obtendo o maior número de assinaturas de usuários e receita total\todo{citar mhealth-market}, é o de atividades físicas (\textit{fitness} em inglês). Tendo em vista este fator, a Jungle Devs, empresa onde foi realizado este trabalho de PFC, foi contratada por um cliente da Austrália para desenvolver um aplicativo móvel envolvendo problemas dentro deste segmento. Em linhas gerais, o projeto tem o objetivo de desenvolver aplicativos para as plataformas \textit{iOS} e \textit{Android} que facilitem a conexão entre academias de musculação e treinadores pessoais profissionais, permitindo que aquelas sejam locadas por estes.

Este trabalho de PFC propõem o desenvolvimento do aplicativo para o sistema \textit{iOS}. Uma versão piloto foi desenvolvida para validar a ideia de negócio proposta para o setor de atividades físicas. O aplicativo tem a função de expor as funcionalidades de negócio para os usuários através de uma interface gráfica. O mesmo também exerce a responsabilidade de integrador dos sistemas que compõe os requisitos da aplicação (sistemas de envio de SMS, geolocalização e pagamento), realizando também a integração com o back-end, que contém as regras de negócio e interage com o banco de dados remoto.

Além do desenvolvimento do aplicativo, este PFC contempla um estudo preliminar dos processos de implementação de testes automatizados dentro da Jungle Devs. A dedicação de esforços voltados ao setor de testes automatizados é uma tendência observada nas empresas de tecnologia modernas. A nível global, cerca de $56\%$ das empresas de tecnologia já estão fazendo tantos ou mais testes automatizados em relação à quantidade de testes manuais \todo{citar tests-trends}. 

% *********
% Objetivos
% *********
\section{Objetivos}
Os objetivos gerais deste PFC foram o desenvolvimento de um aplicativo para o sistema \textit{iOS} e a realização de estudo preliminar da implementação de testes automatizados dentro da empresa. 

O desenvolvimento do aplicativo visa a fase piloto da aplicação que conecta academias de musculação e treinadores. Neste trabalho é apresentado: a metodologia de desenvolvimento do software, a arquitetura de software, a escolha das tecnologias utilizadas, as etapas de desenvolvimento da interface gráfica e de integração de sistemas. 

O estudo dos testes automatizados tem o foco nos testes voltados à interface gráfica, apresentando um comparativo de uso de ferramentas específicas para a tarefa e uma análise do cenário para a adoção de testes dentro da empresa.

% *********
% Estrutura
% *********
\section{Estrutura do Documento}
Este trabalho está estruturado da seguinte forma: 

No capítulo 2, é apresentado o contexto em que o projeto está inserido. A empresa em que este trabalho foi realizado é detalhada, assim como o cenário de uso da aplicação.

No capítulo 3, os conceitos básicos utilizados para a formalização do projeto são expostos. A teoria da metodologia de desenvolvimento de software utilizada no projeto é apresentada, juntamente com a explicação do funcionamento do ambiente de desenvolvimento de aplicativos para o sistema \textit{iOS}, bem como os conceitos técnicos de programação utilizados ao longo do projeto e a teoria na área de testes automatizados.

No capítulo 4, é apresentado o aplicativo da perspectiva do usuário, abordando detalhes pontuais do desenvolvimento quando necessário.

No capítulo 5, são apresentadas as etapas de implementação do aplicativo. Neste capítulo são descritas a metodologia de desenvolvimento, a arquitetura de software, a implementação da interface gráfica, a integração de sistemas e a implementação dos testes automatizados.

No capítulo 6, são apresentados os resultados do desenvolvimento do aplicativo e do estudo da implementação dos testes automatizados. Uma análise do estado final da aplicação é feita, com a descrição do piloto do aplicativo. Também é apresentado um comparativo entre as ferramentas de automatização de testes de interface gráfica utilizadas, bem como uma análise crítica do cenário de uso das mesmas dentro da empresa.

Por fim, no capítulo 7, conclusões e perspectivas sobre o trabalho são levantadas.
