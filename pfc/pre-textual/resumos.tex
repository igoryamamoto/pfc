% ---
% RESUMOS
% ---

% ------------------------------------------------------------------------
% resumo em português
% ------------------------------------------------------------------------

\setlength{\absparsep}{18pt} % ajusta o espaçamento dos parágrafos do resumo
\begin{resumo}
% Descrição geral da empresa (natureza, mercado, processos, etc.), problema-foco atacado no PFC, o que foi feito, principais resultados atingidos, etc.
% Se o documento for escrito em outra língua que não o Português, então é necessário fazer um Resumo Estendido em Português, ao invés deste resumo enxuto.

O PFC foi realizado na Jungle Devs, empresa de desenvolvimento de software voltada à aplicações web e mobile. A mesma foi contratada para desenvolver um aplicativo, objeto de estudo deste trabalho, para locação de academias de musculação na Austrália. O objetivo deste PFC é apresentar um estudo das etapas de desenvolvimento do aplicativo para o sistema \textit{iOS}. Este estudo contempla as seguintes fases do projeto: abordagens e metodologias de desenvolvimento de software (como \textit{Agile} e \textit{Scrum}), planejamento da arquitetura de software (modelo em camadas), escolha de tecnologias (ambiente de desenvolvimento, linguagens de programação e \textit{frameworks}), desenvolvimento da interface gráfica, integração com outros sistemas através de \textit{API's RESTful} (como sistemas de pagamento e de geolocalização). Além disso, durante a etapa final do projeto, foi realizada uma pesquisa de ferramentas emergentes (\textit{EarlGrey}, \textit{Appium} e \textit{XCUITest}) para desenvolvimento de testes automatizados de interface gráfica. Como resultado da pesquisa e uso destas ferramentas, é apresentado um estudo de melhoria de processos da empresa no que tange a etapa de testes. Por fim, o desenvolvimento do aplicativo resultou no lançamento da versão piloto da aplicação, a qual permite que academias para locação sejam cadastradas e que treinadores pessoais agendem horários para treinos com seus clientes.

 \textbf{Palavras-chave}: Software, testes automatizados, aplicativo.
\end{resumo}

% ------------------------------------------------------------------------
% resumo em inglês
% ------------------------------------------------------------------------

\begin{resumo}[Abstract]
 \begin{otherlanguage*}{english}
Resumo em Inglês
   \vspace{\onelineskip}
 
   \noindent 
   \textbf{Keywords}: Software, automated tests.
 \end{otherlanguage*}
\end{resumo}
