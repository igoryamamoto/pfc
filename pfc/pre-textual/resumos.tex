% ---
% RESUMOS
% ---

% ------------------------------------------------------------------------
% resumo em português
% ------------------------------------------------------------------------

\setlength{\absparsep}{18pt} % ajusta o espaçamento dos parágrafos do resumo
\begin{resumo}
% Descrição geral da empresa (natureza, mercado, processos, etc.), problema-foco atacado no PFC, o que foi feito, principais resultados atingidos, etc.
% Se o documento for escrito em outra língua que não o Português, então é necessário fazer um Resumo Estendido em Português, ao invés deste resumo enxuto.

O PFC foi realizado na Jungle Devs, empresa sediada em Florianópolis e fundada em 2018 por ex-alunos do curso de Engenharia de Controle e Automação da UFSC. A empresa desenvolve software para aplicações web e mobile de clientes do exterior e nacionais, além de oferecer programas de capacitação para futuros profissionais de tecnologia.

A Jungle Devs foi contratada para desenvolver um aplicativo para locação de academias de musculação na Austrália, o qual será objeto de estudo deste PFC. A aplicação consiste em uma plataforma onde academias podem ser cadastradas para locação, permitindo que treinadores pessoais agendem horários para treinos com seus clientes. O objetivo deste PFC será o estudo do desenvolvimento do aplicativo para o sistema iOS, contemplando as etapas de planejamento da arquitetura de software, escolha de tecnologias, desenvolvimento da interface gráfica, integração com outros sistemas e implementação de testes.

O estudo do desenvolvimento do aplicativo também será utilizado para melhoria dos processos da empresa no que tange a etapa de testes voltados a interface gráfica. Durante a etapa final do projeto, foi realizada uma pesquisa de ferramentas emergentes para testes automatizados de interface gráfica. Três delas foram utilizadas a nível experimental, resultando em um estudo comparativo.





 \textbf{Palavras-chave}:.
\end{resumo}

% ------------------------------------------------------------------------
% resumo em inglês
% ------------------------------------------------------------------------

\begin{resumo}[Abstract]
 \begin{otherlanguage*}{english}
Resumo em Inglês
   \vspace{\onelineskip}
 
   \noindent 
   \textbf{Keywords}:Palavras Chaves.
 \end{otherlanguage*}
\end{resumo}
